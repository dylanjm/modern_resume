\documentclass[letterpaper]{deedy-resume} 
\usepackage{fixltx2e}
\usepackage{hyperref}
\usepackage{color}
\usepackage{ragged2e}
\usepackage{enumitem}

\begin{document}

%----------------------------------------------------------------------------------------
%	TITLE SECTION
%----------------------------------------------------------------------------------------
\begin{flushleft}
  {\Huge \textbf{Dylan James McDowell}\\}
  \vspace{-3.5mm}
  {\large\textbf{\\ (385) 329-9743 | dylan.mcdowell226@gmail.com | github.com/dylanjm }}
\end{flushleft}


%----------------------------------------------------------------------------------------
%	LEFT COLUMN
% ----------------------------------------------------------------------------------------
\vspace{-3mm}
\hspace{-6.5mm}
\begin{minipage}[t]{0.31\textwidth}
\section{Education} 

\subsection{BYU-Idaho}
\descript{B.S. Applied Mathematics}
\location{December 2018 | Advanced GPA: 3.68}
\vspace{.02cm}
Specialized focus in:
Probability Theory, Machine Learning,
Data Science, and Linear Modeling
\sectionspace

\subsection{BYU-Idaho}
\descript{B.S. Financial Economics}
\location{December 2018 | Advanced GPA: 3.67}
\vspace{.02cm}
Specialized focus in:
Econometrics, Pricing Theory,
Business Valuation, Quantitative Macroeconomics

\section{Skills}
\subsection{Minors}
\textbullet{} Computer Science: 25 Credits \\

\sectionspace

\subsection{Programming}
\location{Proficient}
\textbullet{} R \textbullet{} Python \textbullet{} C++ \\
\textbullet{} Bash \textbullet{} SQL \textbullet{} Java \textbullet{} \LaTeX{} \\ 

\sectionspace

\section{Extra Curricular}
\textbf{Mathematics TA:} Developed an R tutorial video series for the
Mathematics Department to help students and staff learn
how to program using the statistical computation language, R.\\
\sectionspace
\textbf{Computer Science TA:} Assisted students in understanding
Abstract Data Types, Sorting Algorithms, Class Structuring,
Inheritance, and Polymorphism. \\
\sectionspace
\textbf{Economics Society President:} Mentored students
interested in Economics by planning guest speakers relevant in the
field of Economics.\\
\sectionspace
\textbf{Investment Society President:} Managed the
society's \$5.2 million endowment fund.\\


\section{Course work}
\subsection{Undergraduate}
Object Oriented Programming \\
Data Structures \& Algorithms \\
Software Design \& Development \\
Machine Learning \\
Probability \& Statistics \\
Data Visualization \\
Calculus I, II, III \\
Discrete Mathematics \\
Bayesian Statistics \\
Linear Algebra \\
Real Analysis
\sectionspace 

\end{minipage} 
\hfill
%
%----------------------------------------------------------------------------------------
%	RIGHT COLUMN
%----------------------------------------------------------------------------------------
%
\begin{minipage}[t]{0.65\textwidth} 
\section{Professional Experience}

\runsubsection{Computational Science Intern} |\descript{\small Idaho National Laboratory}
\location{Jan 2019 – Present | Idaho Falls, ID}
\vspace{\topsep} 
\begin{tightitemize}
\item Completed major DOE milestone over three months early by migrating all validation cases of the Bison framework over to a new documentation system using \textit{Python, Lisp}, and \textit{Pandoc}. \\
\item Aided in development of NEML, a structural material library, by implementing xml parsing functionality that decreased dependencies on several large C++ libraries.\\
\item Recreated engineering experiments of post-doctoral students by generating finite element mesh models using Cubit and Paraview. \\ 
\item Visualized large scale assessment data using R and Python from the output of the nightly runs on the INL's high performance computing system. \\
\end{tightitemize}

\sectionspace 

\runsubsection{Data Scientist} |\descript{\small Research \& Business Development Center}
\location{April 2018 – December 2018 | Rexburg, ID}
\begin{tightitemize}
  \item Saved the firm thousands of dollars in labor and validation costs by building a full-stack data entry software that automated statistical \& experimental analysis including visualizations using \textit{Shiny, R, SQL}, and \textit{Javascript}. \\
  \item Automated the statistical analysis of 15+ client projects using R to check assumptions and run multi-block ANOVA and Tukey Test on agronomy data.  \\
\end{tightitemize}

\sectionspace 

\runsubsection{Data Science Lab Manager} |\descript{\small BYU-Idaho Tutoring Services}
\location{September 2017 – December 2018 | Rexburg, ID}
\begin{tightitemize}
  \item Instructed students using R to complete assignments across many different subjects. \\
  \item Taught students how to view analytical problems by encouraging them to utilize the tidyverse methodology of R. \\
\end{tightitemize}

\sectionspace 

\runsubsection{Summer Analyst - Private Equity} |\descript{\small Goldman, Sachs \& Co.}
\location{June 2017 – August 2017 | Salt Lake City, UT}
\begin{tightitemize}
  \item Generated cash flow analysis, distribution, and fund performance reports.\\
  \item Automated several weekly deliverables using Python and Excel VBA. \\
  \item Forecast revenues to the alternative investments business using regression analysis. \\
  \item Engineered several programs in VBA, R, and Python to assist analyst in large projects.
\end{tightitemize}

\sectionspace 

\runsubsection{Statistical Analyst} |\descript{\small BYU-Idaho Statistical Consulting Group}
\location{January 2017 – April 2017 | Rexburg, ID}
\begin{tightitemize}
\item Modeled student foot traffic and crowd behavior in a geospatial analysis using R. \\
\item Created an interactive user dashboard for all campus businesses using R-Shiny. \\
\item Programmed several scripts in R to retrieve, clean, and visualize data.
\end{tightitemize}

\sectionspace
    
\section{Research Experience}

\descript{Visual Altimeter using Convolutional Neural Nets}
\location{Fall 2018 | Brigham Young University - Idaho, Dept. of Mathematics}
\begin{tightitemize}
\item Worked with a professor and client to develop a CNN to determine height above the ground using a drone image as input. \\
\item Simulated data using Microsoft's AirSim to gather training data at different heights and then used Python and Keras to train a CNN to classify heights based on input picture with 87\% accuracy.\\
\end{tightitemize}
\sectionspace 

\descript{Analyzing Future Success of Gap Year Students}
\location{Winter 2017 | Brigham Young University - Idaho, Dept. of Economics}
\begin{tightitemize}
\item Assisted professor in post-doctoral research on the future success of students who chose to take a gap year after high school before going to college using a Dynamic-Factor Hidden Markov Model. \\
\item Programmed a web-scraping application in Python that was able to find a list of the top colleges in United States going back to 2002.\\ 
\end{tightitemize}
\sectionspace 

\end{minipage}

\end{document}
% Local Variables:
% TeX-engine: xetex
% TeX-master: t
% End:
